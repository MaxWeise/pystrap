\documentclass[a4paper, 12pt]{scrartcl}

\usepackage[utf8]{inputenc}
\usepackage[T1]{fontenc}
\usepackage[english]{babel}
\usepackage{xcolor}
\usepackage{color}
\usepackage{tabu}
\usepackage{helvet}
\usepackage{listings}
\usepackage{hyperref}


\definecolor{dkgreen}{rgb}{0,0.6,0}
\definecolor{gray}{rgb}{0.5,0.5,0.5}
\definecolor{mauve}{rgb}{0.58,0,0.82}

% Farben in Lstlistings verwenden
\lstset{frame=tb,
    language=Python,
    aboveskip=3mm,
    belowskip=3mm,
    showstringspaces=false,
    columns=flexible,
    basicstyle={\small\ttfamily},
    numbers=left,
    keywordstyle=\color{mauve},
    commentstyle=\color{gray},
    stringstyle=\color{dkgreen},
    breaklines=true,
    breakatwhitespace=true,
    tabsize=4
}

\renewcommand{\familydefault}{\sfdefault}

\author{Max Weise}
\title{pystrap - Project Specifications}

\begin{document}
\maketitle
\tableofcontents

\section{File Specifications}
The following section shows how the config files will be created by the
programm. It will also contain a minimal example, which is used as a reference
for developers.

\subsection{pyproject.toml File}
The \texttt{pyproject.toml} file is used to configure the project in a
standarditzed format. We can define dependencies, metadata and even entry
points for console scripts here. Many third party tools use the
\texttt{pyproject.toml} file for configuration, that's why this file is the
default output of the pystrap script. The following
table~\ref{tab:supported_data_fields} shows a list of all supported fields and
the default values asociated with them.

\begin{table}[!ht]
    \begin{tabu}{| X[l] | X[2,l] | X[1.4,l] | X[l] |}
        \hline
        \textbf{Field} & \textbf{Description} & \textbf{Type} & \textbf{Default Value} \\
        \hline
        [project] & Nonoptional section-identifier. Define metadata of the project here. & Top-level section & - \\ \hline
        name & The name of the project. & String & - \\ \hline
        version & The version of the project. pystrap uses semantic versioning as default scheme & String & 0.0.1 \\ \hline
        authors & List of authors of the project. An author consists of a name and an email adress. & List[String, String] & - \\ \hline
        maintainers & List of maintainers of the project. pystrap will fill in the same data as in the authors field. & List[String, String] & - \\ \hline
        requires-python & Specify the minimal version of python the application needs to run on. & String & $\geq$ 3.10 \\ \hline
        [build-system] & Specifies the system used to build the python application. & Top-level section & - \\ \hline
        requires & Define the dependencies for the build system & List[String, String] & ["setuptools $\geq$ 42", "wheel"] \\ \hline
    \end{tabu}
    \caption{Supported data fields.}
    \label{tab:supported_data_fields}
\end{table}

\newpage

\subsection{setup.py File}
The \texttt{setup.py} file can be used as a way to define requirements and
other metadata for a python project. As per
\href{https://peps.python.org/pep-0621/}{PEP 621}, the standardized way of
storing metadata is the \texttt{pyproject.toml} file. \texttt{setup.py} is only
needed, if the project should be distributed to PyPi. Because of this, the file
created by the pystrap script is a relatively empty one and there are no extra
user inputs needed to construct the file. Codelisting~\ref{lst:setup_file}
shows the contents generated by the script.

\begin{lstlisting}[language=Python, caption={The setup.py file}, label={lst:setup_file}]
    from setuptools import setup

    if __name__ == '__main__':
        setup()
\end{lstlisting}

\subsection{package-init Files}
To define package hierarchies python uses init files, which are allways called \\
\verb|__init__.py| and sit in the directory. While it is \emph{technically}
possible to write python code inside of these files that gets executed at
import-time, there are currently no benefits in doing so. Pystrap will only
create files in the apropriated positions, but will not insert any text or
configuration inside of these files. If a user desires to do so, they can
manipulate the generated file.


\end{document}
